\documentclass[a4paper, 12pt]{article}		% general format

%%%% Charset
\usepackage{cmap}							% make PDF files searchable and copyable
\usepackage[utf8]{inputenc}					% accept different input encodings
\usepackage[T2A]{fontenc}					% russian font
\usepackage[russian]{babel}					% multilingual support (T2A)

%%%% Graphics
\usepackage[dvipsnames]{xcolor}			% driver-independent color extensions
\usepackage{graphicx}						% enhanced support for graphics
\usepackage{wrapfig}						% produces figures which text can flow around

%%%% Math
\usepackage{amsmath}						% American Mathematical Society (AMS) math facilities
\usepackage{amsfonts}						% fonts from the AMS
\usepackage{amssymb}						% additional math symbols

%%%% Typograpy (don't forget about cm-super)
\usepackage{microtype}						% subliminal refinements towards typographical perfection
\linespread{1.3}							% line spacing
\usepackage[left=2.5cm, right=1.5cm, top=2.5cm, bottom=2.5cm]{geometry}
\setlength{\parindent}{0pt}					% we don't want any paragraph indentation
\usepackage{parskip}						% not big distance between paragraphs

%%%% Tables
\usepackage{tabularx}						% Normal tables
\usepackage{multirow}						% for tabular
\usepackage{hhline}							% for tabular

%%%% Graph
\usepackage{tikz}
\usetikzlibrary{arrows}

%%%% Other
\usepackage{url}							% verbatim with URL-sensitive line breaks
%------------------------------------------------------------------------------
\usepackage{listings}						% typeset source code listings

% Цвета для кода
\definecolor{mygreen}{HTML}{3F7F5F} 		% color values Red, Green, Blue
\definecolor{mylilas}{RGB}{170,55,241}

% Настройки отображения кода
\lstset{language=Matlab,%
    %basicstyle=\color{red},
    breaklines=true,						% Перенос длинных строк
    morekeywords={matlab2tikz},
    keywordstyle=\color{blue},%
    morekeywords=[2]{1}, keywordstyle=[2]{\color{black}},
    identifierstyle=\color{black},%
    stringstyle=\color{mylilas},
    commentstyle=\color{mygreen},%
    showstringspaces=false,					% don't mark spaces in strings
    frame=tblr								% draw a frame at all sides of the code block
	rulecolor=\color{frame},				% Цвет рамки
	tabsize=2,								% tab space width
	showstringspaces=false,					% don't mark spaces in strings
    numbers=left,%
    numberstyle={\tiny \color{black}},% size of the numbers
    numbersep=9pt, % this defines how far the numbers are from the text
    emph=[1]{for,end,break},emphstyle=[1]\color{red}, %some words to emphasise
    %emph=[2]{word1,word2}, emphstyle=[2]{style},    
	% Для отображения русского языка
	extendedchars=true,
	literate={Ö}{{\"O}}1
	 	{Ä}{{\"A}}1
	 	{Ü}{{\"U}}1
		{ß}{{\ss}}1
		{ü}{{\"u}}1
		{ä}{{\"a}}1
		{ö}{{\"o}}1
		{~}{{\textasciitilde}}1
		{а}{{\selectfont\char224}}1
		{б}{{\selectfont\char225}}1
		{в}{{\selectfont\char226}}1
		{г}{{\selectfont\char227}}1
		{д}{{\selectfont\char228}}1
		{е}{{\selectfont\char229}}1
		{ё}{{\"e}}1
		{ж}{{\selectfont\char230}}1
		{з}{{\selectfont\char231}}1
		{и}{{\selectfont\char232}}1
		{й}{{\selectfont\char233}}1
		{к}{{\selectfont\char234}}1
		{л}{{\selectfont\char235}}1
		{м}{{\selectfont\char236}}1
		{н}{{\selectfont\char237}}1
		{о}{{\selectfont\char238}}1
		{п}{{\selectfont\char239}}1
		{р}{{\selectfont\char240}}1
		{с}{{\selectfont\char241}}1
		{т}{{\selectfont\char242}}1
		{у}{{\selectfont\char243}}1
		{ф}{{\selectfont\char244}}1
		{х}{{\selectfont\char245}}1
		{ц}{{\selectfont\char246}}1
		{ч}{{\selectfont\char247}}1
		{ш}{{\selectfont\char248}}1
		{щ}{{\selectfont\char249}}1
		{ъ}{{\selectfont\char250}}1
		{ы}{{\selectfont\char251}}1
		{ь}{{\selectfont\char252}}1
		{э}{{\selectfont\char253}}1
		{ю}{{\selectfont\char254}}1
		{я}{{\selectfont\char255}}1
		{А}{{\selectfont\char192}}1
		{Б}{{\selectfont\char193}}1
		{В}{{\selectfont\char194}}1
		{Г}{{\selectfont\char195}}1
		{Д}{{\selectfont\char196}}1
		{Е}{{\selectfont\char197}}1
		{Ё}{{\"E}}1
		{Ж}{{\selectfont\char198}}1
		{З}{{\selectfont\char199}}1
		{И}{{\selectfont\char200}}1
		{Й}{{\selectfont\char201}}1
		{К}{{\selectfont\char202}}1
		{Л}{{\selectfont\char203}}1
		{М}{{\selectfont\char204}}1
		{Н}{{\selectfont\char205}}1
		{О}{{\selectfont\char206}}1
		{П}{{\selectfont\char207}}1
		{Р}{{\selectfont\char208}}1
		{С}{{\selectfont\char209}}1
		{Т}{{\selectfont\char210}}1
		{У}{{\selectfont\char211}}1
		{Ф}{{\selectfont\char212}}1
		{Х}{{\selectfont\char213}}1
		{Ц}{{\selectfont\char214}}1
		{Ч}{{\selectfont\char215}}1
		{Ш}{{\selectfont\char216}}1
		{Щ}{{\selectfont\char217}}1
		{Ъ}{{\selectfont\char218}}1
		{Ы}{{\selectfont\char219}}1
		{Ь}{{\selectfont\char220}}1
		{Э}{{\selectfont\char221}}1
		{Ю}{{\selectfont\char222}}1
		{Я}{{\selectfont\char223}}1
		{і}{{\selectfont\char105}}1
		{ї}{{\selectfont\char168}}1
		{є}{{\selectfont\char185}}1
		{ґ}{{\selectfont\char160}}1
		{І}{{\selectfont\char73}}1
		{Ї}{{\selectfont\char136}}1
		{Є}{{\selectfont\char153}}1
		{Ґ}{{\selectfont\char128}}1
}

% Для настройки заголовка кода
\usepackage{caption}
\DeclareCaptionFont{white}{\color{сaptiontext}}
\DeclareCaptionFormat{listing}{\parbox{\linewidth}{\colorbox{сaptionbk}{\parbox{\linewidth}{#1#2#3}}\vskip-4pt}}
%\captionsetup[lstlisting]{format=listing,labelfont=white,textfont=white}
\renewcommand{\lstlistingname}{Листинг} % Переименование Listings в нужное именование структуры

%------------------------------------------------------------------------------

\begin{document}

\input{titlepage}
\tableofcontents

%------------------------------------------------------------------------------
%\input{introduction}
\newpage
\section*{Введение}
\addcontentsline{toc}{section}{Введение}

%------------------------------------------------------------------------------

\input{optimization}% Вариант 26

%------------------------------------------------------------------------------

\input{gert} % Вариант 13

%------------------------------------------------------------------------------

\newpage
\section{Поиск оптимальной стратегии принятия решений с использованием марковских моделей.}
% Вариант 26, стр 273

\subsection{Постановка задачи}
Пусть имеется машина (станок), которая обслуживается периодически один раз в час. В каждый момент она может находиться в одном из двух состояний: рабочем (состояние 1) и нерабочем (состояние 2).

Если машина на некотором шаге проработала непрерывно 1 час, то доход равен 3 рублям. При этом вероятность остаться на следующем шаге в состоянии 1 равна 0,7, а вероятность перейти в состояние 2 равна 0,3. Если машина отказала на некотором шаге, то её можно отремонтировать двумя способами. Первый является ускоренным, требует затрат в 2 рубля (доход равен -2 рубля) и обеспечивает переход в состояние 1 с вероятностью в 0,6. Второй, обычный способ требует затрат в 1 рубль и обеспечивает переход в состояние 1 с вероятностью 0,4.

Найти оптимальную стратегию для $N=\infty$ методом итераций по стратегиям, и перечислить все стационарные стратегии; построить марковскую модель принятия решений.

%------------------------------------------------------------------------------

\newpage
\section{Поиск оптимальных параметров сети систем массового обслуживания.}
% Вариант 154, задача 5

\subsection{Постановка задачи}
Минимизировать стоимость ССМО при ограничении на среднее число заявок в сети

\begin{gather*}
min\{F(u) = \sum\limits_{j=1}^n \sum\limits_{k=1}^{n_j} f_{jk} u_{jk} = \sum\limits_{j=1}^n \sum\limits_{k=1}^{n_j} (m_{jk} * \mu_{jk}) * u_{jk} \}
\end{gather*} 

$ L(u) = \sum\limits_{j=1}^n L_{jk} u_{jk}$,

$ u_{jk} =
  \begin{cases}
    0,\\
    1
 \end{cases}$,
 
Дано многоканальная сеть Джексона:

\begin{gather*}
\{ \lambda_0, \{jk\} - \text{набор альтернатив}, Q=\{q_{ij}\}_{i=\overline{0,n}, j=\overline{0,n}}, L(u) \}
\end{gather*} 

$L(u) = 4$, (предельное число заявок в сети)

$\lambda_0 = 6$,

$Q=\{q_{ij}\}_{i=\overline{0,n}, j=\overline{0,n}} = \begin{tabular}{|c|c|c|c|c|}
\hline 
0 & 0,2 & 0,7 & 0 & 0,1 \\ 
\hline 
0,1 & 0 & 0,1 & 0,2 & 0,6 \\ 
\hline 
0,5 & 0,1 & 0 & 0,2 & 0,2 \\ 
\hline 
0 & 0,2 & 0,8 & 0 & 0 \\ 
\hline 
0,3 & 0,2 & 0,2 & 0,3 & 0 \\ 
\hline 
\end{tabular} $.

Самостоятельно сформировать набор альтернатив (по 2 альтернативы на каждый узел, обеспечивающих установивший режим в узле).

Решить задачу 5 двумя способами:

\begin{itemize}
\item В соответствии с алгоритмом 5
\item Как задачу дискретного линейного программирования (например, с использованием Матлабовской команды Linprog).
\end{itemize}

%------------------------------------------------------------------------------

%\input{conclusion}
\newpage
\section*{Заключение}
\addcontentsline{toc}{section}{Заключение}

%------------------------------------------------------------------------------

%\input{literature}
\newpage
\section*{Список используемой литературы}
\addcontentsline{toc}{section}{Список используемой литературы}

1.	Колесников Д.Н., Бендерская Е.Н., Лупин А.В., Пахомова В.И., Сиднев А.Г., Цыган В.Н. «Системный анализ и принятие решений». СПб.: Издательство Политехнический университет, 2008. - 468с.

2.  Филипс и Гарси «Методы анализа сетей»

3.	Макаров И.М., и др. Теория выбора и принятия решений. М. Наука. 1982. — 328 с.

4.	Вишневский В.М. «Теоретические основы проектирования компьютерных сетей». — М. : Техносфера, 2003. - 512 с

%------------------------------------------------------------------------------

\end{document}