\documentclass[a4paper, 12pt]{article}		% general format

%%%% Charset
\usepackage{cmap}							% make PDF files searchable and copyable
\usepackage[utf8x]{inputenc} 				% accept different input encodings
\usepackage[T2A]{fontenc}					% russian font
\usepackage[russian]{babel}					% multilingual support (T2A)

%%%% Graphics
\usepackage[dvipsnames]{xcolor}			% driver-independent color extensions
\usepackage{graphicx}						% enhanced support for graphics
\usepackage{wrapfig}						% produces figures which text can flow around

%%%% Math
\usepackage{amsmath}						% American Mathematical Society (AMS) math facilities
\usepackage{amsfonts}						% fonts from the AMS
\usepackage{amssymb}						% additional math symbols

%%%% Typograpy (don't forget about cm-super)
\usepackage{microtype}						% subliminal refinements towards typographical perfection
\linespread{1.3}							% line spacing
\usepackage[left=2.5cm, right=1.5cm, top=2.5cm, bottom=2.5cm]{geometry}
\setlength{\parindent}{0pt}					% we don't want any paragraph indentation
\usepackage{parskip}						% not big distance between paragraphs

%%%% Tables
\usepackage{tabularx}						% Normal tables
\usepackage{multirow}						% for tabular
\usepackage{hhline}							% for tabular

%%%% Graph
\usepackage{tikz}
\usetikzlibrary{arrows}

%%%% Other
\usepackage{url}							% verbatim with URL-sensitive line breaks
%------------------------------------------------------------------------------
\usepackage{listings}						% typeset source code listings

% Цвета для кода
\definecolor{mygreen}{HTML}{3F7F5F} 		% color values Red, Green, Blue
\definecolor{mylilas}{RGB}{170,55,241}

% Настройки отображения кода
\lstset{language=Matlab,%
    %basicstyle=\color{red},
    breaklines=true,						% Перенос длинных строк
    morekeywords={matlab2tikz},
    keywordstyle=\color{blue},%
    morekeywords=[2]{1}, keywordstyle=[2]{\color{black}},
    identifierstyle=\color{black},%
    stringstyle=\color{mylilas},
    commentstyle=\color{mygreen},%
    showstringspaces=false,					% don't mark spaces in strings
    frame=tblr								% draw a frame at all sides of the code block
	rulecolor=\color{frame},				% Цвет рамки
	tabsize=2,								% tab space width
	showstringspaces=false,					% don't mark spaces in strings
    numbers=left,%
    numberstyle={\tiny \color{black}},% size of the numbers
    numbersep=9pt, % this defines how far the numbers are from the text
    emph=[1]{for,end,break},emphstyle=[1]\color{red}, %some words to emphasise
    %emph=[2]{word1,word2}, emphstyle=[2]{style},    
	% Для отображения русского языка
	extendedchars=true,
	literate={Ö}{{\"O}}1
	 	{Ä}{{\"A}}1
	 	{Ü}{{\"U}}1
		{ß}{{\ss}}1
		{ü}{{\"u}}1
		{ä}{{\"a}}1
		{ö}{{\"o}}1
		{~}{{\textasciitilde}}1
		{а}{{\selectfont\char224}}1
		{б}{{\selectfont\char225}}1
		{в}{{\selectfont\char226}}1
		{г}{{\selectfont\char227}}1
		{д}{{\selectfont\char228}}1
		{е}{{\selectfont\char229}}1
		{ё}{{\"e}}1
		{ж}{{\selectfont\char230}}1
		{з}{{\selectfont\char231}}1
		{и}{{\selectfont\char232}}1
		{й}{{\selectfont\char233}}1
		{к}{{\selectfont\char234}}1
		{л}{{\selectfont\char235}}1
		{м}{{\selectfont\char236}}1
		{н}{{\selectfont\char237}}1
		{о}{{\selectfont\char238}}1
		{п}{{\selectfont\char239}}1
		{р}{{\selectfont\char240}}1
		{с}{{\selectfont\char241}}1
		{т}{{\selectfont\char242}}1
		{у}{{\selectfont\char243}}1
		{ф}{{\selectfont\char244}}1
		{х}{{\selectfont\char245}}1
		{ц}{{\selectfont\char246}}1
		{ч}{{\selectfont\char247}}1
		{ш}{{\selectfont\char248}}1
		{щ}{{\selectfont\char249}}1
		{ъ}{{\selectfont\char250}}1
		{ы}{{\selectfont\char251}}1
		{ь}{{\selectfont\char252}}1
		{э}{{\selectfont\char253}}1
		{ю}{{\selectfont\char254}}1
		{я}{{\selectfont\char255}}1
		{А}{{\selectfont\char192}}1
		{Б}{{\selectfont\char193}}1
		{В}{{\selectfont\char194}}1
		{Г}{{\selectfont\char195}}1
		{Д}{{\selectfont\char196}}1
		{Е}{{\selectfont\char197}}1
		{Ё}{{\"E}}1
		{Ж}{{\selectfont\char198}}1
		{З}{{\selectfont\char199}}1
		{И}{{\selectfont\char200}}1
		{Й}{{\selectfont\char201}}1
		{К}{{\selectfont\char202}}1
		{Л}{{\selectfont\char203}}1
		{М}{{\selectfont\char204}}1
		{Н}{{\selectfont\char205}}1
		{О}{{\selectfont\char206}}1
		{П}{{\selectfont\char207}}1
		{Р}{{\selectfont\char208}}1
		{С}{{\selectfont\char209}}1
		{Т}{{\selectfont\char210}}1
		{У}{{\selectfont\char211}}1
		{Ф}{{\selectfont\char212}}1
		{Х}{{\selectfont\char213}}1
		{Ц}{{\selectfont\char214}}1
		{Ч}{{\selectfont\char215}}1
		{Ш}{{\selectfont\char216}}1
		{Щ}{{\selectfont\char217}}1
		{Ъ}{{\selectfont\char218}}1
		{Ы}{{\selectfont\char219}}1
		{Ь}{{\selectfont\char220}}1
		{Э}{{\selectfont\char221}}1
		{Ю}{{\selectfont\char222}}1
		{Я}{{\selectfont\char223}}1
		{і}{{\selectfont\char105}}1
		{ї}{{\selectfont\char168}}1
		{є}{{\selectfont\char185}}1
		{ґ}{{\selectfont\char160}}1
		{І}{{\selectfont\char73}}1
		{Ї}{{\selectfont\char136}}1
		{Є}{{\selectfont\char153}}1
		{Ґ}{{\selectfont\char128}}1
}

% Для настройки заголовка кода
\usepackage{caption}
\DeclareCaptionFont{white}{\color{сaptiontext}}
\DeclareCaptionFormat{listing}{\parbox{\linewidth}{\colorbox{сaptionbk}{\parbox{\linewidth}{#1#2#3}}\vskip-4pt}}
%\captionsetup[lstlisting]{format=listing,labelfont=white,textfont=white}
\renewcommand{\lstlistingname}{Листинг} % Переименование Listings в нужное именование структуры

%------------------------------------------------------------------------------

\begin{document}

\begin{titlepage}
\thispagestyle{empty}

\begin{center}
Санкт-Петербургский политехнический университет Петра Великого\\
Институт Информационных Технологий и Управления\\*
Кафедра компьютерных систем и программных технологий\\*
\hrulefill
\end{center}

\vspace{15em}

\begin{center}
\textsc{\textbf{Курсовая работа}}
\vspace{1em}

Дисциплина: \textbf{Методы оптимизации}
\vspace{2em}

Тема: \textbf{Формулировка и решение задачи выбора оптимального решения с использованием различных математических моделей}
\end{center}

\vspace{16em}

\begin{flushleft}
Выполнил студент гр. 53501/3 \hrulefill С.А. Мартынов \\
\vspace{1.5em}
Руководитель, к.т.н.,доц. \hrulefill А.Г. Сиднев\\
\end{flushleft}

\vspace{\fill}

\begin{center}
Санкт-Петербург \\
2015
\end{center}

\end{titlepage}
\setcounter{page}{2}
\tableofcontents

%------------------------------------------------------------------------------
%\input{introduction}
\newpage
\section*{Введение}
\addcontentsline{toc}{section}{Введение}

В главе 3 рассматриваются марковские цепи, которые являются одним из основных инструменвтов моделирования стохастических систем. 

Существует два метода решения задачи с бесконечным числом этапов. Первый метод основан на переборе всех возможных стационарных стратегий в задаче при­нятия решений. Этот подход, по существу, эквивалентен методу полного перебора, и его можно использовать только тогда, когда общее число стационарных страте­гий с точки зрения практических вычислений достаточно мало. Второй метод, на­зываемый методом итераций по стратегиям, как правило, более эффективен, так как определяет оптимальную стратегию итерационным путем.
%------------------------------------------------------------------------------

\newpage
\section{Варианты формализации многокритериальной задачи и их решение с использованием Optimization Toolbox  системы Matlab.}


\subsection{Постановка задачи}
Мебельная  фабрика выпускает столы, стулья, бюро и книжные шкафы. При изготовлении используются два типа досок, причем фабрика имеет в наличии 1500 м досок первого типа и 1000 м досок второго типа. Кроме того, заданы трудовые ресурсы в количестве 800 чел/час. В таблице приводятся нормативы затрат каждого из видов ресурсов на изготовление 1 ед изделия и прибыль от реализации 1  ед  изделия.

\begin{table}[htb]
	\begin{tabularx}{\textwidth}{|X|c|c|c|c|}
	\hline 
	\multirow{2}{*}{Ресурсы} & \multicolumn{4}{c|}{Затраты на 1 ед изделия} \\ 
	\hhline{~----}
	{} & столы & стулья & бюро & Книжные шкафы \\ 
	\hline 
	Доски первого типа, м & 5 & 1 & 9 & 12 \\ 
	\hline 
	Доски второго типа, м & 2 & 3 & 4 & 1 \\ 
	\hline 
	Трудовые ресурсы, чел/час & 3 & 2 & 5 & 10 \\ 
	\hline 
	Прибыль, руб/шт & 12 & 5 & 15 & 10 \\ 
	\hline 
	\end{tabularx} 
\caption{Нормативы затрат ресурсов на единицу изделия}
\end{table}

По этим исходным данным решить задачу определения оптимальный ассортимент, максимизирующий прибыль (разность между выручкой и расходами.) и выручку при следующих ценах изготавливаемую мебель:

\begin{itemize}
\item стол -- 32 руб;
\item стул -- 15 руб;
\item бюро -- 12 руб;
\item книжный шкаф -- 80 руб.
\end{itemize}

В отчёте необходимо описать:
\begin{enumerate}
	\item Осуществление перехода от многокритериальной задачи к однокритериальной с использованием различных подходов.
	\item Решение задачи стохастического программирования для одной из однокритериальных задач, превратив детерминированное ограничение в вероятностное по схеме:
	$P(\sum\limits_{j=1}^n a_{ij}k_j-b_j\leq0)\geq\alpha_i$
	
	Менять $\alpha_i$ в следующем диапазоне $0.1 \leq \alpha_i \leq 0.9$.
	
	Считать случайной величиной $b_i$ или элементы $\{a_{ij}\}$ $i$-й строки матрицы $A$ $\{a_{ij}\}$ (по выбору).
\end{enumerate}

\subsection{Выделение главного критерия}
Выбирается один из критериев, например $C_i$, который наиболее полно отражает цель принятия решений. Остальные критерии учитываются только с точки зрения возможного указания их нижних границ $C_j(a) \geq \gamma_i$, $ j\neq i$. Таким образом, исходная задача многокритериального принятия решений заменяется однокритериальной задачей с критерием $C_i$, т.е. $a^* = \text{arg max } C_i(a)$, при ограничениях $C_k (a) \geq \gamma_i$, $k\neq i$.

Критерии:
\begin{itemize}
\item $max (12x_1+5x_2+15x_3+10x_4)$ (прибыль)
\item $max (32x_1+15x_2+12x_3+80x_4)$ (выручка)
\end{itemize}

Ограничения:
\begin{itemize}
\item $5x_1+x_2+9x_3+12x_4 \leq 1500$ (доски первого типа)
\item $2x_1+3x_2+4x_3+1x_4 \leq 1000$ (доски второго типа)
\item $3x_1+2x_2+5x_3+1x_4 \leq 800$ (трудовые ресурсы)
\end{itemize}

\subsubsection{Максимизация выручки}

Целевая функция:

$f = min (-32x_1-15x_2-12x_3-80x_4)$

Начальные условия:

$x_0 =
\begin{pmatrix}
  0 \\
  0 \\
  0 \\
  0
\end{pmatrix}$

Ограничения:

$A =
\begin{pmatrix}
  5 & 1 & 9 & 12 \\
  2 & 3 & 4 & 1 \\
  3 & 2 & 5 & 1 \\
  -1& 0 & 0 & 0 \\
  0 &-1 & 0 & 0 \\
  0 & 0 &-1 & 0 \\
  0 & 0 & 0 & -1
\end{pmatrix}$

$b =
\begin{pmatrix}
  1500 \\
  1000 \\
  800 \\
  0 \\
  0 \\
  0 \\
  0
\end{pmatrix}$

\lstinputlisting[language=Matlab, caption={Поиск оптимального решения для максимизация выручки}]{../task1/max_gain.m}

Результат:
\begin{itemize}
\item $x_1 = -0,0000$
\item $x_2 = 300,0000$
\item $x_3 = 0$
\item $x_4 = 100,0000$
\item $f1 = -2500$
\item $f2 = -12500$
\end{itemize}

\subsubsection{Максимизация прибыли}

Целевая функция:

$f = min (-12x_1-5x_2-15x_3-10x_4)$

Начальные условия:

$x_0 =
\begin{pmatrix}
  0 \\
  0 \\
  0 \\
  0
\end{pmatrix}$

Ограничения:

$A =
\begin{pmatrix}
  5 & 1 & 9 & 12 \\
  2 & 3 & 4 & 1 \\
  3 & 2 & 5 & 1 \\
  -1& 0 & 0 & 0 \\
  0 &-1 & 0 & 0 \\
  0 & 0 &-1 & 0 \\
  0 & 0 & 0 & -1 \\
  -32 & -15 & -12 & -80
\end{pmatrix}$

$b =
\begin{pmatrix}
  1500 \\
  1000 \\
  800 \\
  0 \\
  0 \\
  0 \\
  0 \\
  -12500
\end{pmatrix}$

\newpage
\lstinputlisting[language=Matlab, caption={Поиск оптимального решения для максимизация прибыли}]{../task1/max_profit.m}

Результат:
\begin{itemize}
\item $x_1 = 261,2903$
\item $x_2 = 0$
\item $x_3 = 0,0000$
\item $x_4 = 16,1290$
\item $f1 = -3296,8$
\item $f2 = -9651,6$
\end{itemize}

\subsection{Свертка критериев}

Максимизируется критерий объединенной операции, получающийся в результате суммирования всех частных критериев:

$C(a)=\sum\limits_{i=1}^m w_i C_i^n (a)$

$C_i^n (a)=\frac{C_i (a)}{C_i^*}$

$C_i^*$ - оптимальное решение задачи по каждому критерию в отдельности, $w_1+w_2+\dots+w_m=1$.

\lstinputlisting[language=Matlab, caption={Свертка критериев}]{../task1/convolution.m}

В fmincon передается сумма нормированных значений (первый критерий делится на f1, второй на f2), каждое из которых умножено на определенный весовой коэффициент. Результат:
\begin{itemize}
\item $x_1 = 166,4573$
\item $x_2 = 127,8185$
\item $x_3 = -0,0000$
\item $x_4 = 44,9913$
\item $f = -0,9019$ (суммарное)
\end{itemize}

\subsection{Максимин или минимакс}

Максиминную свертку представим в следующем виде: $C_i(a)= \text{min } w_i C_i(a)$

Решение $a^*$ является наилучшим, если для всех $a$ выполняется условие $C(a^*) \geq C(a)$, или $a^* = \text{arg max } C(a) = \text{arg max min } w_i C_i (a)$.

Решение задачи представлено как программа в среде Matlab, с использованием функции fminimax:

$f_1=((12x_1+5x_2+15x_3+10x_4)/3214)^{-1}$;

$f_2=((32x_1+15x_2+12x_3+80x_4 )^2/12500)^{-1}$;

\lstinputlisting[language=Matlab, caption={Содержание файла maxmin.m}]{../task1/maxmin.m}

\newpage
\lstinputlisting[language=Matlab, caption={Содержание файла funminmax.m}]{../task1/funminmax.m}

Так как в среде Matlab реализована только функция fminimax, которая минимизирует наихудшие значения системы функций нескольких переменных, начиная со стартовой оценки ($x_0$), то для реализации максиминной свертки необходимо в fminimax передавать функции, возведенные в степень "-1" (функция funminmax).

Результат:
\begin{itemize}
\item $x_1 = 111,6707$
\item $x_2 = 201,6612$
\item $x_3 = -0,0000$
\item $x_4 = 61,6654$
\item $f1 = 1,0840$
\item $f2 = 1,0840$
\end{itemize}

\subsection{Метод последовательных уступок}

Для решения данной задачи была выбрана уступка = 10\%. Решение задачи представлено как программа в среде Matlab, с использованием функции fmincon.

Целевые функции:
\begin{itemize}
\item $f_1=-(12x_1+5x_2+15x_3+10x_4)$
\item $f_2=-(32x_1+15x_2+12x_3+80x_4)$
\end{itemize}

Для первого критерия:

$A =
\begin{pmatrix}
  5 & 1 & 9 & 12 \\
  2 & 3 & 4 & 1 \\
  3 & 2 & 5 & 1 \\
  -1& 0 & 0 & 0 \\
  0 &-1 & 0 & 0 \\
  0 & 0 &-1 & 0 \\
  0 & 0 & 0 & -1
\end{pmatrix}$

$b =
\begin{pmatrix}
  1500 \\
  1000 \\
  800 \\
  0 \\
  0 \\
  0 \\
  0
\end{pmatrix}$

Результат:
\begin{itemize}
\item $x_1 = 261,29$
\item $x_2 = 0$
\item $x_3 = 0$
\item $x_4 = 16,13$
\item $f1 = 3297$
\item $f2 = 9651$
\end{itemize}

3297 – 329,7 = 2967,3  (10\%)

Для второго критерия:

$A =
\begin{pmatrix}
  5 & 1 & 9 & 12 \\
  2 & 3 & 4 & 1 \\
  3 & 2 & 5 & 1 \\
  -1& 0 & 0 & 0 \\
  0 &-1 & 0 & 0 \\
  0 & 0 &-1 & 0 \\
  0 & 0 & 0 & -1 \\
  -5 & -1 & -9 & -12
\end{pmatrix}$

$b =
\begin{pmatrix}
  1500 \\
  1000 \\
  800 \\
  0 \\
  0 \\
  0 \\
  0 \\
  -2967,3
\end{pmatrix}$

Результат:
\begin{itemize}
\item $x_1 = 112,7$
\item $x_2 = 200,3$
\item $x_3 = 0$
\item $x_4 = 61,4$
\item $f1 = 2967$
\item $f2 = 11519$
\end{itemize}

\subsection{Fgoalattain}

fgoalattain решает задачу достижения цели, которая является одной из формулировок задач для векторной оптимизации.

x = fgoalattain(fun, $x_0$, goal, weight):
\begin{itemize}
\item fun -- целевая функция, 
\item $х_0$ -- начальные значения,
\item goal -- целевые значения,
\item weight -- веса.
\end{itemize}

Решение задачи представлено как программа в среде Matlab, с использованием функций fminicon и fgoalattain.

Целевые значения:

Goal =(15855000 10240038400036 68000000 38080000 4900000) 

\newpage
Веса:

weight=abs(goal) -- для того, чтобы приближение к критериям было одинаково

\lstinputlisting[language=Matlab, caption={Содержание файла fgoalattain.m}]{../task1/fgoalattain.m}

Результат:
\begin{itemize}
\item $x_1 = 207,81$
\item $x_2 = 72,08$
\item $x_3 = 0$
\item $x_4 = 32,4$
\item $f1 = 3178$
\item $f2 = 10324$
\item $Att. = 0,1741$
\end{itemize}

\subsection{Задача стохастического программирования}

Требуется найти такие $x_1$, $x_2$, $x_3$, $x_4$ для которых выполняться следующие ограничения:
\begin{itemize}
\item $5x_1+x_2+9x_3+12x_4 \leq 1500$
\item $2x_1+3x_2+4x_3+1x_4 \leq 1000$
\item $3x_1+2x_2+5x_3+1x_4 \leq 800$
\end{itemize}

Перейдем от последнего ограничения к вероятностному по схеме:
$P(\sum\limits_{j=1}^n a_{ij}k_j-b_j\leq0)\geq\alpha_i$
	
И будем менять $\alpha_i$ в диапазоне $0.1 \leq \alpha_i \leq 0.9$, и возьмём коэффициенты $a_i$ за случайные величины.

$P(0,6x_1 + 0,8x_2 + 1,0x_3 + 1,2x_4 \leq 100) \geq \alpha_i$

Пользуясь формулой:
$\sum\limits_{j=1}^3 a_{ij} x_j - b + K_\alpha \sigma_A \leq 0$

получим вероятностное ограничение для задачи, где  a = {0,6, 0,8, 1,0, 1,2}, b = 100 -- взяты из первоначального вида ограничения, $\sigma_A = \sqrt{x \text{ cov(a) } x^T}$.

По таблице функции распределения стандартного нормального закона находим коэффициенты $K_\alpha (0,5 \leq  \alpha \leq  0,9)$:
\begin{itemize}
\item $K_{0,5} = 0$
\item $K_{0,6} = 0,253$
\item $K_{0,7} = 0,520$
\item $K_{0,8} = 0,841$
\item $K_{0,9} = 1,282$
\end{itemize}

\lstinputlisting[language=Matlab]{../task1/st1.m}
\lstinputlisting[language=Matlab]{../task1/st2.m}
\lstinputlisting[language=Matlab]{../task1/st3.m}

\begin{table}[htb]
	\begin{tabularx}{\textwidth}{|X|X|X|X|X|X|}
	\hline 
	\multirow{2}{*}{} & \multicolumn{5}{c|}{K} \\ 
	\hhline{~-----}
	{} & 0 & 0,253 & 0,52 & 0,841 & 1,282 \\ 
	\hline 
	$x_1$ & 100 & 29,2780 & 27,7214 & 30,0887 & 25,4234 \\ 
	\hline 
	$x_2$ & 0 & 28,7760 & 25,8883 & 26,9406 & 22,6038 \\ 
	\hline 
	$x_3$ & 40 & 28,4840 & 24,8369 & 25,4863 & 21,2327 \\ 
	\hline 
	$x_4$ & 0 & 15,4300 & 12,8728 & 0,5164 & 1,1430 \\ 
	\hline 
	$f$ & 1800 & 1076,8 & 963,4 & 883,2 & 748 \\
	\hline 
	\end{tabularx} 
\caption{Результаты}
\end{table}

Видно, что задача чувствительна к выбранному ограничению, т.к. для различных K получились разные результаты. Так же следует отметить, что значения функций соответствуют нормальному закону распределения, что соответствует теории.% Вариант 26

%------------------------------------------------------------------------------

\newpage
\section{Решение задачи оценки показателей эффективности стохастической сети с использованием методики GERT. Выбор и использование математического пакета Matlab для решения сформулированной задачи.}

\subsection{Постановка задачи}
Дано:
\begin{enumerate}
	\item {Граф GERT-сети (рисунок 1).

	\begin{figure}[!h]
	\centering
	\begin{tikzpicture}[->,>=stealth',shorten >=1pt,auto,node distance=4cm,
	  thick,main node/.style={circle,fill=blue!20,draw,font=\sffamily\Large\bfseries}]

	  \node[main node] (1) {1};
	  \node[main node] (2) [above right of=1] {2};
	  \node[main node] (3) [below of=1] {3};
	  \node[main node] (4) [below of=2] {4};
	  \node[main node] (5) [below right of=2] {5};
	  \node[main node] (6) [below of=5] {6};

	  \path[every node/.style={font=\sffamily\small}]
		(1)	edge node [left] {0,5} (2)
			edge node [left] {0,5} (4)
		(2)	edge node [left] {} (5)
		(3)	edge node [left] {0,6} (1)
			edge [loop below] node {0,4} (3)
		(4)	edge node [left] {0,1} (2)
			edge node [left] {0,3} (3)
			edge node [left] {0,2} (5)
			edge node [left] {0,4} (6)
		(5)	edge node [left] {} (6);
	\end{tikzpicture}
	\caption{Граф GERT-сети.}
	\end{figure}
}	
	
	\item Каждой дуге-работе $(i,j)$ поставлены в соответствие следующие данные:
	\begin{enumerate}
		\item Закон распределения времени выполнения работы. Будем считать его нормальным.
		\item Параметры закона распределения (математическое ожидание $M$ и дисперсия $D$).
		\item Вероятность $P_{ij}$ выполнения работы, показанная на графе.
	\end{enumerate}
\end{enumerate}

\begin{table}[htb]
\centering
	\begin{tabular}{|c|c|c|c|}
	\hline 
	Начальная вершина & Конечная вершина & $M$ & $D$ \\ 
	\hline 
	1 & 2 & 12 & 9 \\ 
	\hline 
	1 & 4 & 28 & 16 \\ 
	\hline 
	2 & 5 & 14 & 9 \\ 
	\hline 
	3 & 1 & 11 & 4 \\ 
	\hline 
	3 & 3 & 33 & 16 \\ 
	\hline 
	4 & 2 & 11 & 4 \\ 
	\hline 
	4 & 3 & 33 & 25 \\ 
	\hline 
	4 & 5 & 45 & 25 \\ 
	\hline 
	4 & 6 & 23 & 16 \\ 
	\hline 
	5 & 6 & 43 & 25 \\ 
	\hline 
	\end{tabular} 
\caption{Параметры закона распределения для дуг графа}
\end{table}

Найти:
\begin{enumerate}
	\item Вероятность выхода в завершающий узел графа (для всех вариантов узел 6).
	\item Математическое ожидание.
	\item Дисперсию времени выхода процесса в завершающий узел графа.
\end{enumerate}

Перечислить все петли всех порядков, обнаруженные на графе, выписать уравнение Мейсона, получить решение для $W_E(s)$ и найти требуемые параметры. Примерно так, как это сделано в примере на стр. 403 -- 409 книги Филипса и Гарсиа «Методы анализа сетей»

\subsection{Ход работы}
Решение:

Замкнём граф дугой из вершины 6 в вершину 1 (рисунок 2).

\begin{figure}[!h]
	\centering
	\begin{tikzpicture}[->,>=stealth',shorten >=1pt,auto,node distance=4cm,
	  thick,main node/.style={circle,fill=blue!20,draw,font=\sffamily\Large\bfseries}]

	  \node[main node] (1) {1};
	  \node[main node] (2) [above right of=1] {2};
	  \node[main node] (3) [below of=1] {3};
	  \node[main node] (4) [below of=2] {4};
	  \node[main node] (5) [below right of=2] {5};
	  \node[main node] (6) [below of=5] {6};

	  \path[every node/.style={font=\sffamily\small}]
		(1)	edge node [left] {0,5} (2)
			edge node [left] {0,5} (4)
		(2)	edge node [left] {} (5)
		(3)	edge node [left] {0,6} (1)
			edge [loop below] node {0,4} (3)
		(4)	edge node [left] {0,1} (2)
			edge node [left] {0,3} (3)
			edge node [left] {0,2} (5)
			edge node [left] {0,4} (6)
		(5)	edge node [left] {} (6)
		(6)	edge [red, bend left] node [left] {$\frac{1}{W_E}$} (1);
	\end{tikzpicture}
	\caption{Замкнутый граф GERT-сети.}
\end{figure}

Петли первого порядка:
\begin{itemize}
\item $W_{12}W_{25}W_{56} \frac{1}{W_E}$
\item $W_{14}W_{42}W_{25}W_{56} \frac{1}{W_E}$
\item $W_{14}W_{43}W_{31}$
\item $W_{14}W_{45}W_{56} \frac{1}{W_E}$
\item $W_{14}W_{46} \frac{1}{W_E}$
\item $W_{33}$
\end{itemize}

Петли второго порядка:
\begin{itemize}
\item $W_{33}$ и $W_{12}W_{25}W_{56} \frac{1}{W_E}$
\item $W_{33}$ и $W_{14}W_{42}W_{25}W_{56} \frac{1}{W_E}$
\item $W_{33}$ и $W_{14}W_{45}W_{56} \frac{1}{W_E}$
\item $W_{33}$ и $W_{14}W_{46} \frac{1}{W_E}$
\end{itemize}

Петель третьего порядка нет.

Выпишем уравнение Мейсона:
\begin{eqnarray*}
	H = 1
		& - & W_{12}W_{25}W_{56} \frac{1}{W_E} \\
		& - & W_{14}W_{42}W_{25}W_{56} \frac{1}{W_E} \\
		& - & W_{14}W_{43}W_{31} \\
		& - & W_{14}W_{45}W_{56} \frac{1}{W_E} \\
		& - & W_{14}W_{46} \frac{1}{W_E} \\
		& - & W_{33} \\
		& + & W_{33}W_{12}W_{25}W_{56} \frac{1}{W_E} \\
		& + & W_{33}W_{14}W_{42}W_{25}W_{56} \frac{1}{W_E} \\
		& + & W_{33}W_{14}W_{45}W_{56} \frac{1}{W_E} \\
		& + & W_{33}W_{14}W_{46} \frac{1}{W_E} = 0
\end{eqnarray*}

Отсюда выведем $W_E(S)$:
\begin{eqnarray*}
	1 - W_{14}W_{43}W_{31} - W_{33} = && (W_{12}W_{25}W_{56} + W_{14}W_{42}W_{25}W_{56} + \\
	&& W_{14}W_{45}W_{56} + W_{14}W_{46} - W_{33}W_{12}W_{25}W_{56} - \\
	&& W_{33}W_{14}W_{42}W_{25}W_{56} - W_{33}W_{14}W_{45}W_{56} - W_{33}W_{14}W_{46})\frac{1}{W_E}
\end{eqnarray*}

\begin{eqnarray*}
	W_E(S) = && (W_{12}W_{25}W_{56} + W_{14}W_{42}W_{25}W_{56} + W_{14}W_{45}W_{56} + W_{14}W_{46} - \\
	&&W_{33}W_{12}W_{25}W_{56} - W_{33}W_{14}W_{42}W_{25}W_{56} - W_{33}W_{14}W_{45}W_{56} - W_{33}W_{14}W_{46})/ \\
	&&(1 - W_{14}W_{43}W_{31} - W_{33})
\end{eqnarray*}

Далее рассчитаем W-функции дуг.

\begin{table}[htb]
\centering
	\begin{tabular}{|c|c|c|c|c|c|}
	\hline 
	Начальная вершина & Конечная вершина & Вес ребра ($p_{ij}$) & $M$ & $D$ & W-функция \\ 
	\hline 
	1 & 2 & 0,5 & 12 & 9 & $0,5*exp(12s+\frac{9}{2}s^2)$ \\ 
	\hline 
	1 & 4 & 0,5 & 28 & 16 & $0,5*exp(28s+\frac{16}{2}s^2)$ \\ 
	\hline 
	2 & 5 & 1 & 14 & 9 & $exp(14s+\frac{9}{2}s^2)$ \\ 
	\hline 
	3 & 1 & 0,6 & 11 & 4 & $0,6*exp(11s+\frac{4}{2}s^2)$ \\ 
	\hline 
	3 & 3 & 0,4 & 33 & 16 & $0,4*exp(33s+\frac{16}{2}s^2)$ \\ 
	\hline 
	4 & 2 & 0,1 & 11 & 4 & $0,1*exp(11s+\frac{4}{2}s^2)$ \\ 
	\hline 
	4 & 3 & 0,3 & 33 & 25 & $0,3*exp(33s+\frac{25}{2}s^2)$ \\ 
	\hline 
	4 & 5 & 0,2 & 45 & 25 & $0,2*exp(45s+\frac{25}{2}s^2)$ \\ 
	\hline 
	4 & 6 & 0,4 & 23 & 16 & $0,4*exp(23s+\frac{16}{2}s^2)$ \\ 
	\hline 
	5 & 6 & 1 & 43 & 25 & $exp(43s+\frac{25}{2}s^2)$ \\ 
	\hline 
	\end{tabular} 
\caption{Производящие функции моментов}
\end{table}

Далее вычислим математическое ожидание и дисперсию: $M_E(s) = 1$ при $s=0$

Поскольку $W_E(s)=p_E M_E (s)$,  то  $p_E=W_E(0)$,  откуда следует, что $M_E(s)=\frac{W_E(s)}{p_E} =\frac{W_E(s)}{W_E(0)}$

Вычисляя первую и вторую производные по $s$ функции $M_E(s)$, и полагая $s=0$, находим математическое ожидание:
$\mu_{1E}=\frac{\partial M_E(s)}{\partial s}|s=0$

и дисперсию:

$\sigma^2=\mu_{2E}-[\mu_{1E}]^2$.

Вероятность выхода в завершающий узел графа:

$p_E=W_E (0)$.

Для моделирования работы был написан скрипт, представленный в листинге 7.

\lstinputlisting[language=Matlab, caption={Код для вычисления заданных выражений}]{../task2/main.m}

\newpage

Результаты работы представлены в листинге 8.

\lstinputlisting[language={},caption={Результат работы скрипта}]{../task2/main.out}

\subsection{Результат}

По итогам проведённых расчётов были получены следующие результаты:
\begin{enumerate}
	\item Вероятность выхода в завершающий узел графа равна 100\% ($p=W_E=1$).
	\item Математическое ожидание 88,46.
	\item Дисперсия времени выхода процесса в завершающий узел графа 2472,3.
\end{enumerate}
 % Вариант 13

%------------------------------------------------------------------------------

\newpage
\section{Поиск оптимальной стратегии принятия решений с использованием марковских моделей.}

\subsection{Постановка задачи}
Пусть имеется машина (станок), которая обслуживается периодически один раз в час. В каждый момент она может находиться в одном из двух состояний: рабочем (состояние 1) и нерабочем (состояние 2).

Если машина на некотором шаге проработала непрерывно 1 час, то доход равен 3 рублям. При этом вероятность остаться на следующем шаге в состоянии 1 равна 0,7, а вероятность перейти в состояние 2 равна 0,3. Если машина отказала на некотором шаге, то её можно отремонтировать двумя способами. Первый является ускоренным, требует затрат в 2 рубля (доход равен -2 рубля) и обеспечивает переход в состояние 1 с вероятностью в 0,6. Второй, обычный способ требует затрат в 1 рубль и обеспечивает переход в состояние 1 с вероятностью 0,4.

Найти оптимальную стратегию для $N=\infty$ методом итераций по стратегиям, и перечислить все стационарные стратегии; построить марковскую модель принятия решений.

\subsection{Марковская модель принятия решений}

Матрицы переходных вероятностей ($P_1$ и $P_2$) и матрицы доходов ($r_1$ и $r_2$) имеют следующий вид:

\[ P_1 = \left( \begin{array}{cc}
0,7 & 0,3 \\
0,6 & 0,4
\end{array} \right)\qquad
%
P_2 = \left( \begin{array}{cc}
0,7 & 0,3 \\
0,4 & 0,6
\end{array} \right)
\]

\[ r_1 = \left( \begin{array}{rr}
3 & 0 \\
-2 & 0
\end{array} \right)\qquad
%
r_2 = \left( \begin{array}{rr}
3 & 0 \\
-1 & 0
\end{array} \right)
\]

Модель представлена на рисунке 3.

\begin{figure}[!h]
	\centering
	\begin{tikzpicture}[->,>=stealth',shorten >=1pt,auto,node distance=4cm,
	  thick,main node/.style={circle,fill=blue!20,draw,font=\sffamily\Large\bfseries}]

	  \node[main node] (1) {$S_1$};
	  \node[main node] (2) [below left of=1] {$D_1$};
	  \node[main node] (3) [below right of=2] {$S_2$};
	  \node[main node] (4) [below right of=1] {$D_2$};

	  \path[every node/.style={font=\sffamily\small}]
	    (1) edge node {0,3} (3)
	        edge [loop above] node {0,7} (1)
	    (2) edge [bend left] node [right] {0,6} (1)
	        edge node[right] {0,4} (3)
	    (3) edge [red, bend left] node [left] {2} (2)
	        edge [red, bend right] node[right] {1} (4)
	    (4) edge node [left] {0,6} (3)
	        edge [bend right] node[right] {0,4} (1);
	\end{tikzpicture}
	\caption{$S_1$ и $S_2$ состояния системы; $D_1$ и $D_2$ принимаемые решения; красные рёбра - траты, чёрные - вероятность перехода}
\end{figure}

После работы, машина можем:
\begin{itemize}
\item Остаться в исправном состоянии
$f^1 = \langle 1; 1 \rangle$
\item Перейти в неисправное состояние
$f^2 = \langle 1; 2 \rangle$
\end{itemize}

Таким образом, возможны следующие стационарные стратегии:
\begin{eqnarray*}
\pi_1^N = (f^1, f^1)\\
\pi_2^N = (f^1, f^2)\\
\pi_3^N = (f^2, f^1)\\
\pi_4^N = (f^2, f^2)
\end{eqnarray*}

\subsection{Метод итерации по стратегиям}

\textbf{Этап оценивания параметров}. Выбираем произвольную стратегию $\tau = (X_{j1}, X_{j2}, \dots, _{jm})^T$. Используя соответствующие стратегии $\tau$, матрицу переходных вероятностей $P(\tau) = (p_{ik}(\tau))$ и матрицу доходов $R(\tau) = (r_{jk}(\tau))$ и полагая $F_\tau(m) = 0$, решаем систему линейных алгебраических уравнений $E_\tau + F_\tau(j) - \sum\limits_{k=1}^m p_{jk}(\tau)F_\tau (k) = v_j (\tau)$, $j=\overline{1,m}$, относительно $E_\tau, F_\tau(1), \dots, F_\tau(m-1)$.

\textbf{Этап улучшения стратегии}. Для каждого состояния $S_j$, находим допустимое решение $X_{*j}$, на котором достигается $\text{max}_{(X_i \in G)}(v_j (X_i) + \sum\limits_{k=1}^m p_{jk}(X_i) F_\tau(k))$

Эти оптимальные решения образуют новую стратегию $t = (X_{*1}, X_{*2}, … X_{*m})^T$. Если $t = \tau$, то стратегия $\tau$ и является оптимальной. В противном случае нужно обозначить стратегию t через $\tau$ и вернуться к первому этапу.

Воспользовавшись матрицами $P_1$, $P_2$, $r_1$, $r_2$ и их независимостью от номера этапа, вычислим ожидаемые доходы, при различных вариантах допустимых решений:

\begin{eqnarray*}
v_1 (X_1 )=0,7*3 + 0,3*0=2,1		\\
v_2 (X_1 )=0,6*(-2) + 0,4*0=-1,2	\\
v_1 (X_2 )=0,7*3 + 0,3*0=2,1		\\
v_2 (X_2 )=0,4*(-1) + 0,6*0=-0,4
\end{eqnarray*}

В качестве произвольной стратегии $\tau$ используем стратегию номер два. В этом случае на этапе оценивания параметров, учитывая, что $F_\tau(2)=0$, получаем систему линейных алгебраических уравнений

$$
\left\{
	\begin{aligned}
	E_\tau + (1 - 0,7) F_\tau(1) &= & 2,1\\
	E_\tau - 0,6 &= & -1,2\\
	\end{aligned}
\right.
$$

которая имеет единственное решение: $E_\tau = 0,78$, $F_\tau(1) = 3,3$.

Результаты соответствующих вычислений приведены в табл. 5.

\begin{table}[htb]
	\begin{tabularx}{\textwidth}{|c|X|X|c|c|}
	\hline
	\multirow{2}{*}{$S_j$} & \multicolumn{2}{c|}{$\varphi (X_i )=v_j (X_k )+p_j1 (X_i)*F_i (1)$} & \multirow{2}{*}{max $\varphi_j$} & \multirow{2}{*}{$X_{*j}$} \\ 
	\hhline{~--~~}
	{} & i = 1 & i = 2 & {} & {} \\ 
	\hline 
	1 & 2,1+0,7*3,3=4,41 & 2,1+0,7*3,3=4,41 & 4,41 & $X_2$ \\ 
	\hline 
	2 & -1,2+0,6*3,3=0,78 & -0,4+0,4*3,3=0,92 & 0,78 & $X_2$ \\ 
	\hline
	\end{tabularx}
\caption{Решение с оценочными параметрами}
\end{table}

Новая стратегия $t = (X_2, X_2)^T$ отличается от стратегии $\tau$, поэтому нужно на этап оценивания параметров, полагая $\tau = (X_2, X_1)^T$.

Новой стратегии t соответствует следующая система линейных алгебраических уравнений:
$$
\left\{
	\begin{aligned}
	E_\tau + (1 - 0,7) F_\tau(1) &= & 2,1\\
	E_\tau - 0,4 &= & -0,4\\
	\end{aligned}
\right.
$$

которая имеет единственное решение: $E_\tau = 0,85$, $F_\tau(1) = 3,125$.

Результаты соответствующих вычислений приведены в табл. 6.

\begin{table}[htb]
	\begin{tabularx}{\textwidth}{|c|X|X|c|c|}
	\hline
	\multirow{2}{*}{$S_j$} & \multicolumn{2}{c|}{$\varphi (X_i )=v_j (X_k )+p_j1 (X_i)*F_i (1)$} & \multirow{2}{*}{max $\varphi_j$} & \multirow{2}{*}{$X_{*j}$} \\ 
	\hhline{~--~~}
	{} & i = 1 & i = 2 & {} & {} \\
	\hline
	1 & 2,1+0,7*3,125=4,2875 & 2,1+0,7*3,125=4,2875 & 4,2875 & $X_2$ \\
	\hline
	2 & -1,2+0,6*3,125=0,675 & -0,4+0,4*3,125=0,85 & 0,85 & $X_2$ \\
	\hline
	\end{tabularx}
\caption{Проверочное решение}
\end{table}

Новая стратегия совпала с предыдщей, таким образом оптимальная стратегия определена: оптимальнее использовать более дешевый ремонт с меньшей гарантией успешного завершения.

\subsection{Метод линейного программирования}

Все параметры посчитаны, мы можем сформулировать задачу в виде задачи линейного программирования, для проверки ранее полученных результатов.

$$
\left\{
	\begin{aligned}
	2,1w_{11}+2,1w_{12}-1,2w_{21}-0,4w_{22} & ->& \text{max}\\
	0,3w_{11}+0,3w_{12}-0,6w_{21}-0,4w_{22} & = & 0 \\
	-0,3w_{11}-0.3w_{12}+0,6w_{21}+0,4w_{22} & = & 0 \\
	\sum\limits_{j=1}^2 \sum\limits_{i=1}^2 w_{ij}=1,\qquad w_{ij} \geq 0 &&\\
	\end{aligned}
\right.
$$

Код скрипта представлен в листинге 9.

\lstinputlisting[language=Matlab, caption={Код для вычисления задачи линейного программирования}]{../task3/linear.m}

\newpage

Результат выполнения в листинге 10.

\lstinputlisting[language={},caption={Результат работы скрипта линейного программирования}]{../task3/linear.out}

Таким образом, оптимальной стратегие снова стало использование дешевого ремонта, как и в предыдущем случае с итерации по стратегиям. % Вариант 26, стр 273

%------------------------------------------------------------------------------

\newpage
\section{Поиск оптимальных параметров сети систем массового обслуживания.}
% Вариант 154, задача 5

\subsection{Постановка задачи}
Минимизировать стоимость ССМО при ограничении на среднее число заявок в сети

\begin{gather*}
min\{F(u) = \sum\limits_{j=1}^n \sum\limits_{k=1}^{n_j} f_{jk} u_{jk} = \sum\limits_{j=1}^n \sum\limits_{k=1}^{n_j} (m_{jk} * \mu_{jk}) * u_{jk} \}
\end{gather*} 

$ L(u) = \sum\limits_{j=1}^n L_{jk} u_{jk}$,

$ u_{jk} =
  \begin{cases}
    0,\\
    1
 \end{cases}$,
 
Дано многоканальная сеть Джексона:

\begin{gather*}
\{ \lambda_0, \{jk\} - \text{набор альтернатив}, Q=\{q_{ij}\}_{i=\overline{0,n}, j=\overline{0,n}}, L(u) \}
\end{gather*} 

$L(u) = 4$, (предельное число заявок в сети)

$\lambda_0 = 6$,

$Q=\{q_{ij}\}_{i=\overline{0,n}, j=\overline{0,n}} = \begin{tabular}{|c|c|c|c|c|}
\hline 
0 & 0,2 & 0,7 & 0 & 0,1 \\ 
\hline 
0,1 & 0 & 0,1 & 0,2 & 0,6 \\ 
\hline 
0,5 & 0,1 & 0 & 0,2 & 0,2 \\ 
\hline 
0 & 0,2 & 0,8 & 0 & 0 \\ 
\hline 
0,3 & 0,2 & 0,2 & 0,3 & 0 \\ 
\hline 
\end{tabular} $.

Самостоятельно сформировать набор альтернатив (по 2 альтернативы на каждый узел, обеспечивающих установивший режим в узле).

Решить задачу 5 двумя способами:

\begin{itemize}
\item В соответствии с алгоритмом 5
\item Как задачу дискретного линейного программирования (например, с использованием Матлабовской команды Linprog).
\end{itemize}

%------------------------------------------------------------------------------

%\input{conclusion}
\newpage
\section*{Заключение}
\addcontentsline{toc}{section}{Заключение}

%------------------------------------------------------------------------------

%\input{literature}
\newpage
\section*{Список используемой литературы}
\addcontentsline{toc}{section}{Список используемой литературы}

1.	Колесников Д.Н., Бендерская Е.Н., Лупин А.В., Пахомова В.И., Сиднев А.Г., Цыган В.Н. «Системный анализ и принятие решений». СПб.: Издательство Политехнический университет, 2008. - 468с.

2.  Филипс и Гарси «Методы анализа сетей»

3.	Макаров И.М., и др. Теория выбора и принятия решений. М. Наука. 1982. — 328 с.

4.	Вишневский В.М. «Теоретические основы проектирования компьютерных сетей». — М. : Техносфера, 2003. - 512 с

%------------------------------------------------------------------------------

\end{document}