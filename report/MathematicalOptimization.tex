\documentclass[a4paper, 12pt]{article}		% general format

%%%% Charset
\usepackage{cmap}							% make PDF files searchable and copyable
\usepackage[utf8]{inputenc}					% accept different input encodings
\usepackage[T2A]{fontenc}					% russian font
\usepackage[russian]{babel}					% multilingual support (T2A)

%%%% Graphics
\usepackage[dvipsnames]{xcolor}			% driver-independent color extensions
\usepackage{graphicx}						% enhanced support for graphics
\usepackage{wrapfig}						% produces figures which text can flow around

%%%% Math
\usepackage{amsmath}						% American Mathematical Society (AMS) math facilities
\usepackage{amsfonts}						% fonts from the AMS
\usepackage{amssymb}						% additional math symbols

%%%% Typograpy (don't forget about cm-super)
\usepackage{microtype}						% subliminal refinements towards typographical perfection
\linespread{1.3}							% line spacing
\usepackage[left=2.5cm, right=1.5cm, top=2.5cm, bottom=2.5cm]{geometry}
\setlength{\parindent}{0pt}					% we don't want any paragraph indentation
\usepackage{parskip}						% not big distance between paragraphs

%%%% Tables
\usepackage{tabularx}						% Normal tables
\usepackage{multirow}						% for tabular
\usepackage{hhline}							% for tabular

%%%% Graph
\usepackage{tikz}
\usetikzlibrary{arrows}

%%%% Other
\usepackage{url}							% verbatim with URL-sensitive line breaks
%------------------------------------------------------------------------------

\begin{document}

\input{titlepage}
\tableofcontents

%------------------------------------------------------------------------------
%\input{introduction}
\newpage
\section*{Введение}
\addcontentsline{toc}{section}{Введение}

%------------------------------------------------------------------------------

\newpage
\section{Варианты формализации многокритериальной задачи и их решение с использованием Optimization Toolbox  системы Matlab.}
% Вариант 26

Мебельная  фабрика выпускает столы, стулья, бюро и книжные шкафы. При изготовлении используются два типа досок, причем фабрика имеет в наличии 1500 м досок первого типа и 1000 м досок второго типа. Кроме того, заданы трудовые ресурсы в количестве 800 чел/час. В таблице приводятся нормативы затрат каждого из видов ресурсов на изготовление 1 ед изделия и прибыль от реализации 1  ед  изделия.

\begin{table}[htb]
	\begin{tabularx}{\textwidth}{|X|c|c|c|c|}
	\hline 
	\multirow{2}{*}{Ресурсы} & \multicolumn{4}{c|}{Затраты на 1 ед изделия} \\ 
	\hhline{~----}
	{} & столы & стулья & бюро & Книжные шкафы \\ 
	\hline 
	Доски первого типа, м & 5 & 1 & 9 & 12 \\ 
	\hline 
	Доски второго типа, м & 2 & 3 & 4 & 1 \\ 
	\hline 
	Трудовые ресурсы, чел/час & 3 & 2 & 5 & 10 \\ 
	\hline 
	Прибыль, руб/шт & 12 & 5 & 15 & 10 \\ 
	\hline 
	\end{tabularx} 
\caption{Нормативы затрат ресурсов на единицу изделия}
\end{table}

По этим исходным данным решить задачу определения оптимальный ассортимент, максимизирующий прибыль и выручку при следующих ценах изготавливаемую мебель:

\begin{itemize}
\item стол -- 32 руб;
\item стул -- 15 руб;
\item бюро -- 12 руб;
\item книжный шкаф -- 80 руб.
\end{itemize}


%------------------------------------------------------------------------------

\newpage
\section{Решение задачи оценки показателей эффективности стохастической сети с использованием методики GERT. Выбор и использование математического пакета Matlab для решения сформулированной задачи.}
% Вариант 13

Дано:
\begin{enumerate}
	\item {Граф GERT-сети.

	\begin{figure}[!h]
	\centering
	\begin{tikzpicture}[->,>=stealth',shorten >=1pt,auto,node distance=4cm,
	  thick,main node/.style={circle,fill=blue!20,draw,font=\sffamily\Large\bfseries}]

	  \node[main node] (1) {1};
	  \node[main node] (2) [above right of=1] {2};
	  \node[main node] (3) [below of=1] {3};
	  \node[main node] (4) [below of=2] {4};
	  \node[main node] (5) [below right of=2] {5};
	  \node[main node] (6) [below of=5] {6};

	  \path[every node/.style={font=\sffamily\small}]
		(1)	edge node [left] {0,5} (2)
			edge node [left] {0,5} (4)
		(2)	edge node [left] {} (5)
		(3)	edge node [left] {0,6} (1)
			edge [loop below] node {0,4} (3)
		(4)	edge node [left] {0,1} (2)
			edge node [left] {0,3} (3)
			edge node [left] {0,2} (5)
			edge node [left] {0,4} (6)
		(5)	edge node [left] {} (6);
	\end{tikzpicture}
	\end{figure}
}	
	
	\item Каждой дуге-работе $(i,j)$ поставлены в соответствие следующие данные:
	\begin{enumerate}
		\item Закон распределения времени выполнения работы. Будем считать его нормальным.
		\item Параметры закона распределения (математическое ожидание $M$ и дисперсия $D$).
		\item Вероятность $P_{ij}$ выполнения работы, показанная на графе.
	\end{enumerate}
\end{enumerate}

\begin{table}[htb]
\centering
	\begin{tabular}{|c|c|c|c|}
	\hline 
	Начальная вершина & Конечная вершина & $M$ & $D$ \\ 
	\hline 
	1 & 2 & 12 & 9 \\ 
	\hline 
	1 & 4 & 28 & 16 \\ 
	\hline 
	2 & 5 & 14 & 9 \\ 
	\hline 
	3 & 1 & 11 & 4 \\ 
	\hline 
	3 & 3 & 33 & 16 \\ 
	\hline 
	4 & 2 & 11 & 4 \\ 
	\hline 
	4 & 3 & 33 & 25 \\ 
	\hline 
	4 & 5 & 45 & 25 \\ 
	\hline 
	4 & 6 & 23 & 16 \\ 
	\hline 
	5 & 6 & 43 & 25 \\ 
	\hline 
	\end{tabular} 
\caption{Параметры закона распределения для дуг графа}
\end{table}

Найти:
\begin{enumerate}
	\item Вероятность выхода в завершающий узел графа (для всех вариантов узел 6).
	\item Математическое ожидания.
	\item Дисперсию времени выхода процесса в завершающий узел графа.
\end{enumerate}

Перечислить все петли всех порядков, обнаруженные на графе, выписать уравнение Мейсона, получить решение для $W_E(s)$ и найти требуемые параметры. Примерно так, как это сделано в примере на стр. 403 -- 409 книги Филипса и Гарсиа «Методы анализа сетей»

%------------------------------------------------------------------------------

\newpage
\section{Поиск оптимальной стратегии принятия решений с использованием марковских моделей.}
% Вариант 26

Пусть имеется машина (станок), которая обслуживается периодически один раз в час. В каждый момент она может находиться в одном из двух состояний: рабочем (состояние 1) и нерабочем (состояние 2). Если машина на некотором шаге проработала непрерывно 1 час, то доход равен 3 рублям. При этом вероятность остаться на следующем шаге в состоянии 1 равна 0,7, а вероятность перейти в состояние 2 равна 0,3. Если машина отказала на некотором шаге, то её можно отремонтировать двумя способами. Первый является ускоренным, требует затрат в 2 рубля (доход равен -2 рубля) и обеспечивает переход в состояние 1 с вероятностью в 0,6. Второй, обычный способ требует затрат в 1 рубль и обеспечивает переход в состояние 1 с вероятностью 0,4.

Найти оптимальную стратегию для $N=\infty$ методом итераций по стратегиям, и перечислить все стационарные стратегии; построить марковскую модель принятия решений.

% стр 273

%------------------------------------------------------------------------------

\newpage
\section{Поиск оптимальных параметров сети систем массового обслуживания.}
% Вариант 154, задача 5

Минимизировать стоимость ССМО при ограничении на среднее число заявок в сети

\begin{gather*}
min\{F(u) = \sum\limits_{j=1}^n \sum\limits_{k=1}^{n_j} f_{jk} u_{jk} = \sum\limits_{j=1}^n \sum\limits_{k=1}^{n_j} (m_{jk} * \mu_{jk}) * u_{jk} \}
\end{gather*} 

$ L(u) = \sum\limits_{j=1}^n L_{jk} u_{jk}$,

$ u_{jk} =
  \begin{cases}
    0,\\
    1
 \end{cases}$,
 
Дано многоканальная сеть Джексона:

\begin{gather*}
\{ \lambda_0, \{jk\} - \text{набор альтернатив}, Q=\{q_{ij}\}_{i=\overline{0,n}, j=\overline{0,n}}, L(u) \}
\end{gather*} 

$L(u) = 4$, (предельное число заявок в сети)

$\lambda_0 = 6$,

$Q=\{q_{ij}\}_{i=\overline{0,n}, j=\overline{0,n}} = \begin{tabular}{|c|c|c|c|c|}
\hline 
0 & 0,2 & 0,7 & 0 & 0,1 \\ 
\hline 
0,1 & 0 & 0,1 & 0,2 & 0,6 \\ 
\hline 
0,5 & 0,1 & 0 & 0,2 & 0,2 \\ 
\hline 
0 & 0,2 & 0,8 & 0 & 0 \\ 
\hline 
0,3 & 0,2 & 0,2 & 0,3 & 0 \\ 
\hline 
\end{tabular} $.

Самостоятельно сформировать набор альтернатив (по 2 альтернативы на каждый узел, обеспечивающих установивший режим в узле).

Решить задачу 5 двумя способами:

\begin{itemize}
\item В соответствии с алгоритмом 5
\item Как задачу дискретного линейного программирования (например, с использованием Матлабовской команды Linprog).
\end{itemize}

%------------------------------------------------------------------------------

%\input{conclusion}
\newpage
\section*{Заключение}
\addcontentsline{toc}{section}{Заключение}

%------------------------------------------------------------------------------

%\input{literature}
\newpage
\section*{Список используемой литературы}
\addcontentsline{toc}{section}{Список используемой литературы}

1.  Филипс и Гарси «Методы анализа сетей»

2. Макаров И.М., и др. Теория выбора и принятия решений. М. Наука. 1982. — 328 с.

%------------------------------------------------------------------------------

\end{document}